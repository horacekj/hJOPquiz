\documentclass[12pt,a4paper]{article}
\usepackage[czech]{babel}
\usepackage[utf8]{inputenc}
\usepackage{graphicx}
\usepackage{enumitem}
\textwidth 16cm \textheight 24.6cm
\topmargin -1.3cm
\oddsidemargin 0cm
\begin{document}
\thispagestyle{empty}
\noindent

\title{
\Large Ovládání kolejiště pomocí hJOP\\
\LARGE Sada otázek\\
\small v1.3}
\author{Jan Horáček (jan.horacek@kmz-brno.cz)}
\date{\today}
\maketitle

\section*{S1 – dispečer}

\begin{enumerate}[leftmargin=*]
\item Co je to závěr a k~čemu slouží?

\item Za jakých podmínek není nutné žádat o traťový souhlas?

\item S~kolika druhy traťových zabezpečovacích zařízení (TZZ) se můžete na
modulovce TT setkat? Popište základní rozdíly mezi jednotlivými typy TZZ.

\item Lze přestavit výhybku, která je obsazená? Jak?

\item Co je to riziková funkce?

\item Které všechny úkony je nutné vykonat při potvrzování rizikové funkce?
Uveďte na příkladu nouzového stavění výhybky.

\item Kde se dozvím, co stanice vyváží a dováží?

\item Které úkony je nutné vykonat k~řádnému zpracování manipulačního vlaku,
který vám právě přijel do stanice? Zpracování manipulačního vlaku končí
v~momentě jeho odjezdu ze stanice.

\item Jak potlačím zvukovou notifikaci zkratu na kolejišti? Je toto potlačení
trvalé?

\item Sousední stanice mě žádá o~traťový souhlas, já ho však chci přijmout až
za minutu, jak nejlépe sdělit sousední stanici, že má počkat?

\item Mohu vypnout napájecí zdroj stanice za plného provozu? Co po vypnutí
napájení udělá ovládací panel?

\item Jak poznám, že je panel připojen k~serveru?

\item Mohu z~libovolného pracoviště ovládat libovolnou stanici?

\item Potřebuji lokomotivní zálohu ze Ždánické výtopny, jak ji získám?

\item Jakými všemi metodami mohu komunikovat s~dispečery sousedních stanic?

\item Jako posunovač ve Ždánicích mám volnou chvíli, ohlédnu se do Klobouk a
vidím, že dva vlaky jedou proti sobě a nezpomalují, přibližně za 5 s~do sebe
narazí. Co udělám?

\item Za jakých podmínek lze nouzově zastavit všechny lokomotivy na kolejišti?
Jak takové zastavení provedu?

\item Jak mohu zrušit jízdní cestu? Popište co nejvíce metod.

\item Popište vztah mezi pojmy \textit{jízdní cesta}, \textit{vlaková cesta} a
\textit{posunová cesta}.

\item Vyjmenujte všechny možnosti, jak řídit lokomotivu ručně.

\item Jak zjistím název lokomotivy na přijíždějícím vlaku v~trati?

\item Kolik lokomotiv může mít souprava?

\item Mohu při restartu systému nechat vlaky v~trati?

\item Je nutné měnit orientaci stanoviště A~při otáčení vlaku v~koncové
stanici?

\item Co je to úvazka?

\item Je nutné zadávat vlaku jeho délku? Proč?

\item Jaký význam má barva dopravní kanceláře na panelu? Popiš významy
jednotlivých barev.

\item Jak poznám dopravní a manipulační kolej?

\item Jak poznám směr trati na reliéfu?

\item Jaké směry trati znáš?

\item Jak zavřu zobrazené menu?

\item Jak zastavím vykolejený vlak v~trati?

\item Co znamená, když je v~regulátoru zaškrtnuto \textit{totální ruční řízení}
a k~čemu mě to opravňuje?

\item Co to znamená, že se na kolejišti \textit{jezdí nákladní doprava} a jak
nákladní doprava funguje?

\item Uveď 3 typy vozů, které jsou pro Ždáňku typické.

\item Kdo odpovídá za vlaky ve stanici?

\item Lze vypravit vlak proti STŮJ?

\item Jak lze přesunout vlak na jinou kolej?

\item Mohu přesunout vlak na libovolný úsek?

\item Co to je bloková podmínka?

\item Uveď alespoň 4 typy bloků.

\item Úvazka je červená, co to znamená?

\item Kam píšu nákladní vozy ve vlaku?

\item Co je to \textit{uLI-daemon} a~jakou má funkci?

\item Co je to \textit{uLI-master} a~jakou má funkci?

\item Jak převezmu lokomotivy do ručního řízení skrze \textit{uLI-daemon}?

\item Jak uvolním lokomotivu z~Rocomaus připojené skrze \textit{uLI-daemon}?

\item Ikonka mašinky na Rocomaus bliká, co to znamená?

\item Ikonka mašinky na Rocomaus trvale svítí, mašinka přesto nereaguje, co je
špatně?

\item Co indikuje modrá LED dioda na \textit{uLI-master}?

\item Chci se přihlásit ke stanici pouze na dívání, jak to provedu?

\item Ke kolejišti přijde můj kamarád, který si chce zajezdit, ale nemá
zaškolení (nemá login), co mu mohu povolit a co naopak nesmím? Jak se k~němu
mám chovat?

\item Jaký je smysl těchto zkoušek?

\end{enumerate}

\newpage
\section*{S2 – pokročilý dispečer}

\begin{enumerate}[leftmargin=*]
\item Chci vjet s~vlakem od vjezdového staničního návěstidla na obsazenou
kolej, jak to udělám?

\item K~čemu slouží přivolávací návěst?

\item K~čemu je dobrý zásobník povelů?

\item Jaký je rozdíl mezí přímou volbou a volbou do zásobníku?

\item Které všechny povely mohu vkládat do zásobníku povelů?

\item Jak smažu povel ze zásobníku povelů?

\item Vysvětli rozdíl mezi nouzovou cestou a přivolávací návěstí.

\item Chci vjet s~vlakem na manipulační kolej, za jakých podmínek a jak to mohu
provést?

\item Zásobník je přepnutý do volby \textit{PV}, je zásobník aktivní?

\item Jak přidám vlaku další lokomotivu?

\item Jak funguje multitrakce v~ručních ovladačích?

\end{enumerate}

\end{document}

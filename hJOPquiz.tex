\documentclass[12pt,a4paper]{article}
\usepackage[czech]{babel}
\usepackage[utf8]{inputenc}
\usepackage{graphicx}
\usepackage{enumitem}
\usepackage{titling}
\usepackage{url}
\textwidth 16cm \textheight 24.6cm
\topmargin -1.3cm
\oddsidemargin 0cm
\begin{document}
\thispagestyle{empty}
\noindent

\setlength{\droptitle}{-5em}

\title{
\Large Ovládání kolejiště pomocí hJOP\\
\LARGE Sada otázek\\
\small v1.3}
\author{Otázky: Jan Horáček (jan.horacek@kmz-brno.cz)\\Odpovědi: Vladan Kudláč (vladan.kudlac@kmz-brno.cz)}
\date{\today}
\maketitle

\section*{S1 – dispečer}

\begin{enumerate}[leftmargin=*]
\item Co je to závěr a k~čemu slouží?

\item Za jakých podmínek není nutné žádat o traťový souhlas?

\item \textit{S~kolika druhy traťových zabezpečovacích zařízení (TZZ) se můžete na
modulovce TT setkat? Popište základní rozdíly mezi jednotlivými typy TZZ.}

\item Lze přestavit výhybku, která je obsazená? Jak?

\item \textbf{Co je to riziková funkce?}
\\Jedná se o rizikové operace, jako například rušení vlaku, přestavování obsazené výhybky, uvolňování závěru. Je potvrzována potvrzovací sekvencí.

\item \textbf{Které všechny úkony je nutné vykonat při potvrzování rizikové funkce?
Uveďte na příkladu nouzového stavění výhybky.}
\\Je nutné projít potvrzovanou rizikovou funkci a výpis kontrolovaných podmínek. Občas se ve výpisu může objevit čekání na nějakou akci (uzavření přejezdu / přestavení výhybky), v takovém případě je nutné počkat na dokončení těchto úkonů. A poté odsouhlasit seznam kontrolovaných podmínek.

\item \textbf{Kde se dozvím, co stanice vyváží a dováží?}
\\Na každé stanici je výtisk datových listů. Dostupné také online: \url{https://github.com/kmzbrnoI/ds-tt/releases}.

\item \textbf{Které úkony je nutné vykonat k~řádnému zpracování manipulačního vlaku,
který vám právě přijel do stanice? Zpracování manipulačního vlaku končí
v~momentě jeho odjezdu ze stanice.}
\\V editaci vlaku je v poznámce k soupravě uvedeno, jak naložit s jednotlivými vagony. Vyřešit požadavky, které se týkají aktuální stanice. Přidat do soupravy vagony dle poptávky a plánů. Editovat počet vozů, délku vlaku a popis vagónů dle změn.

\item \textbf{Jak potlačím zvukovou notifikaci zkratu na kolejišti? Je toto potlačení
trvalé?}
\\Na horní liště je ikona s reproduktorem a křížkem. Potlačení je na 1 minutu.

\item Sousední stanice mě žádá o~traťový souhlas, já ho však chci přijmout až
za minutu, jak nejlépe sdělit sousední stanici, že má počkat?

\item \textbf{Mohu vypnout napájecí zdroj stanice za plného provozu? Co po vypnutí
napájení udělá ovládací panel?}
\\Ano, je to možné, ale zastaví to provoz v celém napájeném úseku. Na panelu se objeví široké modré podbarvení.

\item Jak poznám, že je panel připojen k~serveru?

\item Mohu z~libovolného pracoviště ovládat libovolnou stanici?

\item Potřebuji lokomotivní zálohu ze Ždánické výtopny, jak ji získám?

\item Jakými všemi metodami mohu komunikovat s~dispečery sousedních stanic?

\item Jako posunovač ve Ždánicích mám volnou chvíli, ohlédnu se do Klobouk a
vidím, že dva vlaky jedou proti sobě a nezpomalují, přibližně za 5 s~do sebe
narazí. Co udělám?

\item Za jakých podmínek lze nouzově zastavit všechny lokomotivy na kolejišti?
Jak takové zastavení provedu?

\item Jak mohu zrušit jízdní cestu? Popište co nejvíce metod.

\item \textbf{Popište vztah mezi pojmy \textit{jízdní cesta}, \textit{vlaková cesta} a
\textit{posunová cesta}.}
\\Jízdní cestou rozumíme cestu pro vlak (vlaková cesta) nebo pro posun (posunová cesta).

\item Vyjmenujte všechny možnosti, jak řídit lokomotivu ručně.

\item Jak zjistím název lokomotivy na přijíždějícím vlaku v~trati?

\item Kolik lokomotiv může mít souprava?

\item \textbf{Mohu při restartu systému nechat vlaky v~trati?}
\\Pokud je to možné, dojet s vlaky do stanic. Pokud to není možné, je vhodné poznačit údaje v editaci vlaku. Po restartu bude nutné smazat původní vlak na serveru, přesunou lokomotivu do stanice, kde se fyzicky nachází, a vlak znovu založit.

\item Je nutné měnit orientaci stanoviště A~při otáčení vlaku v~koncové
stanici?

\item \textit{Co je to úvazka?}

\item Je nutné zadávat vlaku jeho délku? Proč?

\item Jaký význam má barva dopravní kanceláře na panelu? Popiš významy
jednotlivých barev.
\\Růžová (neznámý stav), bílá (vzdálený provoz), šedá (místní provoz), žlutá (výběr vozu pro ruční řízení).

\item Jak poznám dopravní a manipulační kolej?

\item Jak poznám směr trati na reliéfu?

\item Jaké směry trati znáš?

\item Jak zavřu zobrazené menu?

\item Jak zastavím vykolejený vlak v~trati?

\item Co znamená, když je v~regulátoru zaškrtnuto \textit{totální ruční řízení}
a k~čemu mě to opravňuje?

\item Co to znamená, že se na kolejišti \textit{jezdí nákladní doprava} a jak
nákladní doprava funguje?

\item \textbf{Uveď 3 typy vozů, které jsou pro Ždáňku typické.}
\\Vtr, Ztr, Rah.

\item Kdo odpovídá za vlaky ve stanici?

\item \textbf{Lze vypravit vlak proti STŮJ?}
\\Ano, na přivolávací návěst.

\item Jak lze přesunout vlak na jinou kolej?

\item Mohu přesunout vlak na libovolný úsek?
\\Ne, lze to jen na úseky s číslem koleje.

\item \textit{Co to je bloková podmínka?}

\item \textit{Uveď alespoň 4 typy bloků.}

\item \textit{Úvazka je červená, co to znamená?}

\item \textbf{Kam píšu nákladní vozy ve vlaku?}
\\V editaci vlaku do Poznámky k soupravě.

\item \textbf{Co je to \textit{uLI-daemon} a~jakou má funkci?}
\\Je to APLIKACE umožňující funkci uLI-master. Spouští se startem prvního panelu na dispečerském počítači, při nechtěném ukončení lze znovu zapnout ikonkou na horní liště panelu.

\item \textbf{Co je to \textit{uLI-master} a~jakou má funkci?}
\\Je to černá krabička rozměru 7x5 cm, umožňující připojení Roco Multimaus k počítači se spuštěným panelem.

\item \textbf{Jak převezmu lokomotivy do ručního řízení skrze \textit{uLI-daemon}?}
\\Kliknutím na číslo vlaku $\rightarrow$ MAUS vlak -- Převzít vlak na multiMaus (uLI-master) a poté zvolit číslo myši s ruč. nebo zvolit ruční řízení dodatečně. Pokud lokomotiva není součástí vlaku, lze kliknout na dopravní kancelář $\rightarrow$ Loko $\rightarrow$ MAUS loko.

\item \textbf{Jak uvolním lokomotivu z~Rocomaus připojené skrze \textit{uLI-daemon}?}
\\Buď na uLI-daemonu zvolit \uv{Uvolnit}, nebo pomocí Shift+STOP na Rocomyši.

\item \textbf{Ikonka mašinky na Rocomaus bliká, co to znamená?}
\\Lokomotiva není v režimu Ručního řízení. Můžeme ovládat funkce, ale ne jízdu.

\item \textbf{Ikonka mašinky na Rocomaus trvale svítí, mašinka přesto nereaguje, co je
špatně?}
\\Může to znamenat problém s napájením mašinky nebo s adresou mašinky.

\item \textit{Co indikuje modrá LED dioda na \textit{uLI-master}?}

\item Chci se přihlásit ke stanici pouze na dívání, jak to provedu?

\item Ke kolejišti přijde můj kamarád, který si chce zajezdit, ale nemá
zaškolení (nemá login), co mu mohu povolit a co naopak nesmím? Jak se k~němu
mám chovat?

\item Jaký je smysl těchto zkoušek?

\end{enumerate}

\newpage
\section*{S2 – pokročilý dispečer}

\begin{enumerate}[leftmargin=*]
\item Chci vjet s~vlakem od vjezdového staničního návěstidla na obsazenou
kolej, jak to udělám?

\item K~čemu slouží přivolávací návěst?

\item K~čemu je dobrý zásobník povelů?

\item Jaký je rozdíl mezí přímou volbou a volbou do zásobníku?

\item Které všechny povely mohu vkládat do zásobníku povelů?

\item Jak smažu povel ze zásobníku povelů?

\item Vysvětli rozdíl mezi nouzovou cestou a přivolávací návěstí.

\item Chci vjet s~vlakem na manipulační kolej, za jakých podmínek a jak to mohu
provést?

\item Zásobník je přepnutý do volby \textit{PV}, je zásobník aktivní?

\item Jak přidám vlaku další lokomotivu?

\item Jak funguje multitrakce v~ručních ovladačích?

\end{enumerate}

\end{document}

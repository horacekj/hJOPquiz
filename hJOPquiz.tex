\documentclass[12pt,a4paper]{article}
\usepackage[unicode,colorlinks=true]{hyperref}
\usepackage[czech]{babel}
\usepackage[utf8]{inputenc}
\usepackage{graphicx}
\textwidth 16cm \textheight 24.6cm
\topmargin -1.3cm 
\oddsidemargin 0cm
\begin{document}
\thispagestyle{empty}
\noindent

\title{
\Large Ovládání kolejiště pomocí hJOP\\
\LARGE Sada otázek}
\author{Jan Horáček (jan.horacek@kmz-brno.cz)}
\date{\today}
\maketitle

\begin{enumerate}
	\item S kolika druhy traťových zabezpečovacích zařízení (TZZ) se můžete na modulovce TT setkat? Popište základní rozdíly mezi jednotlivými typy TZZ.
	
	\item Lze přestavit výhybku, která je obsazená? Jak?
	
	\item Co je to potvrzovací sekvence?
	
	\item Které všechny úkony je nutné vykonat při potvrzování potvrzovací sekvence? Uveďte na konkrétním konkrétní situaci, kterou si sami vymyslíte.
	
	\item Kde se dozvím, co daná stanice vyváží a dováží?
	
	\item Které úkony je nutné vykonat k řádnému zpracování manipulačního vlaku, který vám právě přijel do stanice? Zpracování manipulačního vlaku končí v momentě jeho odjezdu ze stanice.
	
	\item Jak potlačím zvukovou notifikaci zkratu na kolejišti? Je toto potlačení trvalé?
	
	\item Sousední stanice mě žádá o traťový souhlas, já ho však mohu přijmout až za minutu, jak nejlépe sdělit sousední stanici, že má počkat?
	
	\item Chci vjet s vlakem od vjezdového staničního návěstidla na obsazenou kolej, jak to udělám?
	
	\item K čemu slouží přivolávací návěst?
	
	\item Mohu vypnout napájecí zdroj stanice za plného provozu? Co po vypnutí napájení udělá ovládací panel?
	
	\item Jak poznám, že je panel připojen k serveru?
	
	\item Mohu z libovolného pracoviště ovládat libovolnou stanici?
	
	\item Potřebuji lokomotivní zálohu ze Ždánické výtopny, jak ji získám?
	
	\item Jakými všemi metodami mohu komunikovat s dispečery sousedních stanic?
	
	\item Jako posunovač ve Ždánicích mám volnou chvíli, ohlédnu se do Klobouk a vidím, že dva vlaky jedou proti sobě a nezpomalují, přibližně za 5 s do sebe narazí. Co udělám?
	
	\item Za jakých podmínek lze nouzově zastavit všechny lokomotivy na kolejišti? Jak takové zastavení provedu?
	
	\item K čemu je dobrý zásobník povelů?
	
	\item Jaký je rozdíl mezí přímou volbou a volbou do zásobníku?
	
	\item Které všechny povely mohu vkládat do zásobníku povelů?
	
	\item Jak mohu zrušit jízdní cestu? Popište co nejvíce metod.
	
	\item Popište vztah mezi pojmy \textit{jízdní cesta}, \textit{vlaková cesta} a \textit{posunová cesta}.
	
	\item Vyjmenujte všechny možnosti, jak řídit lokomotivu ručně.
	
	\item Který ovládací prvek má přednost: Rocomouse, nebo řídící SW?
	
	\item Na co je potřeba dávat pozor při přebírání lokomotiv na Rocomouse?
	
	\item Jak zjistím název lokomotivy na přijíždějícím vlaku v trati?
	
	\item Kolik lokomotiv může mít souprava?
	
	\item Mohu při restartu systému nechat vlaky v trati?
	
	\item Jak převezmu adresu z Rocomouse zpět do řízení počítače?
	
	\item Je nutné měnit orientaci stanoviště A při otáčení vlaku v koncové stanici?
	
	\item Co je to úvazka?
	
	\item Je nutné zadávat vlaku jeho délku? Proč?
	
	\item Jaký význam má barva dopravní kanceláře na panelu? Popiš významy jednotlivých barev.
	
	\item Jak poznám dopravní a manipulační kolej?
	
	\item Jak poznám směr trati na reliéfu?
	
	\item Jaké směry trati znáš?
	
	\item Jak zavřu zobrazené menu?
	
	\item Jak zastavím vykolejený vlak v trati?
	
	\item Co znamená, když je v regulátoru zaškrtnuto \textit{totální ruční řízení} a k čemu mě to opravňuje?
	
	\item Kterou stanici simuluje smyčka za Krumvíří při zavedené nákladní dopravě?
	
	\item Uveď 3 typy vozů, které se na Ždáňce typicky mohly vyskytovat.
	
	\item Kdo odpovídá za vlaky ve stanici?
	
	\item Lze vypravit vlak proti STŮJ?
	
	\item Vysvětli rozdíl mezi nouzovou cestou a přivolávací návěstí.
	
	\item Jak lze přesunou vlak na jinou kolej?
	
	\item Co to je bloková podmínka?
	
	\item K čemu je přivolávací návěst?
	
	\item Uveď alespoň 4 typy bloků?
	
	\item Úvazka je červená, co to znamená?
	
	\item Kam píšu nákladní vozy ve vlaku?
	
	\item Proveď křižování v Uhřicích.
	
	\item Jak smažu povel ze zásobníku povelů?
	
	
	
\end{enumerate}

\end{document}